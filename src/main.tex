\documentclass[12pt]{article}

\usepackage{sbc-template}

\usepackage{graphicx,url}

%\usepackage[brazil]{babel}   
\usepackage[utf8]{inputenc}  
     
\sloppy

\title{Identifying (deep) fake news communities on multiple social networks}

\author{Thiago Amado Costa\inst{1}, Humberto Torres Marques Neto\inst{1}}


\address{ICEI – Pontifícia Universidade Católica de Minas Gerais (PUC-MG)\\
Belo Horizonte, Minas Gerais - Brasil 
\email{thiago.amado@sga.pucminas.br, humberto@sga.pucminas.br}
}

\begin{document} 

\maketitle

\begin{abstract}
  The misuse of Deepfakes and the rise of Fake News have become a significant challenge in the
  era of social media, posing serious threats to the credibility of information shared online.
  Additionally, individuals and groups can become targets of these malicious actions, creating
  communities that believe and share news of the same malicious sources.
  This problem is not exclusive to a single social media platform, which highlights the urgent
  need for robust fact-checking mechanisms, as well as solutions to identify these specific
  communities, that will help create a safer online environment.
\end{abstract}

\section{Introduction}
% >> Descrição de motivações
% >> Problema escolhido e objetivos

Fake News and Deepfakes are one of the biggest challenges in today's social media, threatening the
credibility of information shared online. 
Fake News is defined as incorrect or misleading information fabricated to mimic the structure of 
the news media, but without the process of ensuring accuracy and credibility (\cite{lazer2018science}),
and Deepfake is a form of multimedia manipulation that leverages advanced machine learning and artificial
intelligence techniques to manipulate or generate visual and audio content, with a high capacity to deceive
viewers (\cite{KIETZMANN2020135}). 

This phenomenon poses serious threats to today's social media, as it spreads faster and more broadly
than credible information (\cite{doi:10.1126/science.aap9559}), and Deepfakes are becoming more
accessible and believable (\cite{KIETZMANN2020135}). This reinforces the importance of further
research on this topic.
Hence, the primary objectives of this study are to analyze various methodologies for detecting 
Fake News and Deepfakes, as well as to examine and identify the communities associated with them.

\bibliographystyle{sbc}
\bibliography{sbc-template}

\end{document}
